\documentclass[a4paper,12pt]{article}
\usepackage[margin=1in]{geometry}
\usepackage[slovene]{babel}
\usepackage[utf8]{inputenc}
\usepackage{lmodern}
\usepackage[T1]{fontenc}
\usepackage{eurosym}
\usepackage{graphicx}
\usepackage{amsfonts}
\usepackage{amssymb}
\usepackage{amsmath}
\usepackage{hyperref}
\usepackage{amsthm}
\usepackage{subfigure}
\usepackage{xcolor}
\usepackage{tcolorbox}
\usepackage{enumitem}


\newcommand{\todo}[1]{{\color{red}{#1}}}
\newcommand{\R}{\mathcal R}
\newcommand{\F}{\mathcal F}
\newcommand{\V}{\mathcal V}

\newtheorem{lema}{Lema}
\newtheorem{izrek}{Izrek}
\newtheorem{opomba}{Opomba}

\begin{document}
\title{Osnovna reprodukcijska števila in podkritična endemična ravnovesna stanja v epidemioloških problemih}
\author{Katarina Černe}
\maketitle

\section{Uvod}

Pri veliko epidemioloških modelih je eno izmed ravnovesnih stanj ravnovesje brez okužbe (DFE). 
Pri analizi teh modelov nam pomembne informacije nudi t.i. osnovno reprodukcijsko 
število \(\R_0\). Če je \(\R_0<1\), je DFE lokalno asimptotsko stabilno, kar pomeni,
da se bolezen ne bo razširila v populaciji. Če pa je \(\R_0>1\), pa je DFE nestabilno
in razširjenje bolezni je možno.

Vseeno pa izpolnjen pogoj \(\R_0<1\) ne zagotavlja nujno, da lahko bolezen izkoreninimo.
Kljub temu, da je pri \(\R_0<1\) DFE lokalno asimptotsko stabilno, lahko v okolici
DFE obstajajo nestabilna podkritična endemična ravnovesna stanja, ki povzročajo, da 
bolezen ostaja prisotna v populaciji. \todo{kaj je res s tem}

V tej seminarski nalogi si ogledamo definicijo osnovnega reprodukcijskega števila 
\(\R_0\) za razdrobljene modele, pokažemo, da \(\R_0\) res nudi podatke o stabilnosti
ravnovesnega stanja brez bolezni DFE, opišemo pogoje za obstoj podkritičnih stanj
in teorijo uporabimo na nekaj primerih.

\section{Osnovno reprodukcijsko število v razdrobljenih epidemioloških problemih} \label{r0}

Najprej si oglejmo splošno strukturo modela populacije, s katerim bomo delali.
Imejmo heterogeno populacijo, ki jo lahko razdelimo v \(n\) homogenih razredov.
Naj vektor \(x=(x_1,x_2,\ldots,x_n)^T\) označuje velikost populacije v vsakem 
izmed \(n\) razredov. Zaradi biološke smiselnosti mora seveda veljati \(x_i\geq 0\),
\(i=1,2,\ldots,n\). Razredi se med seboj ločijo na okužene in neokužene. 
Naj bo prvih \(m\) razredov okuženih, preostali pa neokuženi. Vpeljimo še oznako
\(X_s=\{x\geq 0 | x_i=0, i=1,\ldots, m\}\), to je množica vseh stanj, v katerih
ni bolezni. \todo{nove okužbe}

Sedaj z \(\F_i(x)\) označimo stopnjo pojavitve novih okužb v razredu \(i\). 
Z \(\V^+_i(x)\) označimo stopnjo prehoda v \(i\)-ti razred, ki se ne zgodijo zaradi 
novih okužb temveč iz drugih razlogov, \(\V^-_i\) pa naj označuje stopnjo prehoda
iz \(i\)-tega razreda. 

Model prenosa okužbe lahko zapišemo na naslednji način:
\begin{equation} \label{eq1}
\dot{x}=f_i(x)=\F_i - \V_i,
\end{equation}
kjer je \(\V_i=\V_i^- - \V_i^+\) in \(i=1,\ldots,n\). 
Pisali bomo tudi \(\F=(\F_1,\F_2,\ldots,\F_n)\) in \(\V_1,\V_2,\ldots,\V_n\).

Dobljeni sistem lineariziramo:
\begin{equation} \label{eq2}
\dot{x}=Df(x_0)(x-x_0),
\end{equation}
kjer z \(x_0\) označimo DFE, z \(Df(x_0)\) pa Jacobijevo matriko preslikave 
\(f=(f_i, f_2, \ldots f_n)\).
Za funkcije \(f_i\)
%\(\F_i\), \(\V_i^+\) in \(\V_i^-\) 
morajo veljati še naslednje predpostavke:

\begin{itemize}
    \item[(A1)] \(x\geq 0 \Rightarrow \F_i\textrm{, }\V_i^+\textrm{, }\V_i^- \geq 0\)
    \item[(A2)] \(x_i=0 \Rightarrow \V_i^- =0\), kar pomeni, da prehodi iz praznega 
    razreda niso možni. \todo{v posebnem} 
    \item[(A3)] \(\F_i=0\) za \(i>m\), kar pomeni, da nimamo okužb ve neokuženih razredih - 
    ko se posameznik okuži, preide v okužen razred 
    \item[(A4)] \(x\in X_s \Rightarrow \F_i(x)=0\) in \(\V_i^+(x)=0\) za \(i=1,\ldots,m\)
    To pomeni, da okužba ne pride "od zunaj" temveč samo iz razredov znotraj populacije. 
    \item[(A5)] \(\F(x)=0 \Rightarrow\) vse lastne vrednosti matrike \(Df(x_0)\) imajo 
    negativne realne dele, torej, omejimo se na sisteme, kjer je DFE stabilno, če nimamo novih okužb.    
\end{itemize}

Za lažje delo s prej definiranimi funkcijami si oglejmo naslednjo lemo.

\begin{lema} \label{lema1}
    Naj bo \(x_0\) DFE sistema \ref{eq1} in naj funkcije \(f_i\) zadoščajo predpostavlkam
    (A1)-(A5). Potem sta Jacobijevi matriki za \(\F\) in \(\V\) oblike
    \begin{align*}
        D\F(x_0)= 
        \begin{bmatrix}
        F & 0 \\
        0 & 0 \\
        \end{bmatrix},\textrm{ }
        D\V(x_0)=
        \begin{bmatrix}
        V & 0 \\
        J_3 & J_4 \\
        \end{bmatrix},
    \end{align*}
    kjer sta \(F\) in \(V\) \(m\times m\) matriki, definirani kot
    \begin{align*}
        \big[\frac{\partial \F_i}{\partial x_j}(x_0)\big]_{i,j=1,\ldots m},\textrm{ }
        \big[\frac{\partial \V_i}{\partial x_j}(x_0)\big]_{i,j=1,\ldots m}.
    \end{align*}
    Velja še, da je \(F\) nenegativna, \(V\) je nesingularna M-matrika in vse lastne
    vrednosti matrike \(J_4\) imajo pozitivne realne dele.
\end{lema}

\begin{opomba}
    Pravimo, da je neka matrika M-matrika, če je Z-matrika in imajo njene lastne
    vrednosti nenegativne realne dele. Matriko imenujemo Z-matrika, če so vsi njeni
    izvendiagonalni elementi nepozitivni.
\end{opomba}

\todo{dokaz}

Sedaj, ko imamo definiran model populacije, lahko definiramo osnovno reprodukcijsko
število. Osnovno reprodukcijsko število \(\R_0\) v epidemiološkem modelu interpretiramo
kot pričakovano število novih okužb, ki jih povzroči okuženi osebek v sicer popolnoma
dovzetni populaciji. Če je \(R_0<1\), torej v povprečju okuženi osebek v času svoje
okuženosti ustvari manj kot eno novo okužbo in okužba se zato ne bo razširila. Podobno, če je 
\(R_0>1\), v povprečju okuženi osebek ustvari več kot enega novega okuženega, kar pomeni,
da se okužba lahko razširi po populaciji. 

Matematično lahko v primeru, da imamo samo en razred okuženih, \(\R_0\) definiramo
kot produkt med stopnjo okužbe in pričakovanim časom, ki ga okuženi osebek preživi 
v okuženem razredu. V modelu z več okuženimi razredi pa \(\R_0\) izpeljemo na naslednji način.

Opazovati moramo, kaj se dogaja z okuženim osebkom, ki ga uvedemo v populacijo brez 
bolezni, torej v populacijo v DFE \(x_0\). Ker v začetku še nimamo okužbe, je 
\(D\F(x_0)\) ničeln, torej opazujemo enačbo
\begin{equation}\label{eq3}
    \dot{x}=-D\V(x_0)(x-x_0)
\end{equation}
Sedaj s \(\Psi_i(0)\) označimo število okuženih, ki jih v začetku uvedemo v \(i\)-ti 
razred. Naj bo \(\Psi(t)=(\Psi_1(t),\ldots,\Psi_m(t))^T\) število okuženih ob času \(t\). 
Po lemi \ref{lema1} se da zapisati \(D\V\) kot 
\(
\begin{bmatrix}
    V & 0 \\
    J_3 & J_4 \\
\end{bmatrix}
\),
kjer je \(V\) velikosti \(m \times m\), vektor \(\Psi\) pa v bistvu
predstavlja prvih \(m\) komponent vektorja \(x\), ki reši enačbo \ref{eq3}. Torej 
\(\Psi(t)\) reši enačbo \(\Psi'(t)=-V\Psi(t)\). Ta enačba ima enolično rešitev 
\(\Psi(t)=\exp{-Vt}\Psi(0)\).

Pričakovano število okužb, ki jih ustvarijo okuženi posamezniki v tem primeru,
se izračuna kot \(\int_0^\infty F\Psi(t)dt = \int_0^\infty F\exp{-Vt}\Psi(0)dt = FV^{-1}\Psi(0)\).
Po lemi \ref{lema1} je matrika \(V\) nesingularna, torej \(V^{-1}\) obstaja. 

Sedaj si oglejmo interpretacijo matrike \(FV^{-1}\). Element na \((i,j)\)-tem mestu
v matriki \(F\) predstavlja stopnjo, s katero okuženi osebki iz razreda \(j\)
ustvarijo nove okužbe v razredu \(i\). Element na \((j,k)\)-tem mestu v matriki \(V^{-1}\)
predstavlja pričakovani čas, ki ga nek okuženi posameznik, ki je začel v \(k\)-tem razredu, 
preživi v \(j\)-tem razredu v času svojega življenja. Torej je \((i,k)\)-ti element 
matrike \(FV^{-1}\) pričakovano število novih okužb, ki jih okuženi posameznik, ki 
smo ga na začetki uvedli v \(k\)-ti razred, povzroči v \(i\)-tem razredu.

Osnovno reprodukcijsko število definiramo kot spektralni radij matrike \(FV^{-1}\):
\[\R_0=\rho(FV^{-1}).\]

O povezavi med osnovnim reprodukcijskim številom in lokalno asimptotsko stabilnostjo 
ravnovesnega stanja DFE govori naslednji izrek.

\begin{izrek}
    Imejmo model prenosa bolezni kot v \ref{eq1}, kjer naj za funkcijo \(f\) veljajo
    predpostavke (A1)-(A5). Če je ravnovesno stanje \(x_0\) DFE, potem je \(x_0\) LAS, 
    če velja \(\R_0<1\), in nestabilno, če je \(\R_0>1\), kjer je \(\R_0=\rho(FV^{-1})\).
\end{izrek}

\todo{dokaz}

\section{Podkritična in nadkritična ravnovesna stanja}

Videli smo, da je ravnovesje brez okužbe LAS če je \(\R_0<1\) in nestabilno, če 
je \(\R_0>1\). Težava pa se pojavi v točki bifurkacije, torej če \(\R_0=1\), oziroma
v njeni okolici.

Za lažjo obravnavo dogajanja v okolici točke bifurkacije vpeljimo bifurkacijski parameter
\(\mu\), za katerega naj velja, da je \(\R_0<1\) za \(\mu<0\) in \(\R_0>1\) za \(\mu>0\).

Opazujemo sistem 
\[\dot{x}=f(x,\mu),\]
kjer je \(f\) kot v poglavju \ref{r0} in vsaj dvakrat zvezno odvedljiva v \(x\) in \(\mu\).
Naj ima sistem DFE \(x_0\). Lokalna stabilnost ravnovesja se spremeni v točki \((x_0,0)\), kjer 
imamo torej bifurkacijo. V okolici točke bifurkacije se nam lahko pojavijo endemična 
ravnovesja, ki so lahko superkritična ali subkritična. Superkritična ravnovesja so 
netrivialna \todo{stabilna?} ravnovesja v okolici točke bifurkacije pri \(\R_0>1\),
subkritična pa pri \(\R_0<1\). Slednja lahko pomembno vplivajo na dinamiko v 
epidemioloških modelih, saj lahko že pri majhnih perturbacijah okrog DFE preidemo
v nestabilno ravnovesno stanje in se zato kljub temu, da je \(\R_0<1\), okužba lahko 
razširi v populaciji.

Obstoj 

\section{Primeri}


\end{document}