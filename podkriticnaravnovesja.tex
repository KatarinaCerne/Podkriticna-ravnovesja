\documentclass[a4paper,12pt]{article}
\usepackage[margin=1in]{geometry}
\usepackage[slovene]{babel}
\usepackage[utf8]{inputenc}
\usepackage{lmodern}
\usepackage[T1]{fontenc}
\usepackage{eurosym}
\usepackage{graphicx}
\usepackage{amsfonts}
\usepackage{amssymb}
\usepackage{amsmath}
\usepackage{hyperref}
\usepackage{amsthm}
\usepackage{subfigure}
\usepackage{xcolor}
\usepackage{tcolorbox}
\usepackage{enumitem}


\newcommand{\todo}[1]{{\color{red}{#1}}}
\newcommand{\R}{\mathcal R}
\newcommand{\F}{\mathcal F}
\newcommand{\V}{\mathcal V}

\newtheorem{lema}{Lema}
\newtheorem{izrek}{Izrek}
\newtheorem{opomba}{Opomba}

\begin{document}
\title{Osnovna reprodukcijska števila in podkritična endemična ravnovesna stanja v epidemioloških problemih}
\author{Katarina Černe}
\maketitle

\section{Uvod}

Pri veliko epidemioloških modelih je eno izmed ravnovesnih stanj ravnovesje brez okužbe (DFE). 
Pri analizi teh modelov nam pomembne informacije nudi t.i. osnovno reprodukcijsko 
število \(\R_0\). Če je \(\R_0<1\), je DFE lokalno asimptotsko stabilno, kar pomeni,
da se bolezen ne bo razširila v populaciji. Če pa je \(\R_0>1\), pa je DFE nestabilno
in razširjenje bolezni je možno.

Vseeno pa izpolnjen pogoj \(\R_0<1\) ne zagotavlja nujno, da lahko bolezen izkoreninimo.
Kljub temu, da je pri \(\R_0<1\) DFE lokalno asimptotsko stabilno, lahko v okolici
DFE obstajajo nestabilna podkritična endemična ravnovesna stanja, ki povzročajo, da 
bolezen ostaja prisotna v populaciji. \todo{kaj je res s tem}

V tej seminarski nalogi si ogledamo definicijo osnovnega reprodukcijskega števila 
\(\R_0\) za razdrobljene modele, pokažemo, da \(\R_0\) res nudi podatke o stabilnosti
ravnovesnega stanja brez bolezni DFE, opišemo pogoje za obstoj podkritičnih stanj
in teorijo uporabimo na nekaj primerih.

\section{Osnovno reprodukcijsko število v razdrobljenih epidemioloških problemih} \label{r0}

Najprej si oglejmo splošno strukturo modela populacije, s katerim bomo delali.
Imejmo heterogeno populacijo, ki jo lahko razdelimo v \(n\) homogenih razredov.
Naj vektor \(x=(x_1,x_2,\ldots,x_n)^T\) označuje velikost populacije v vsakem 
izmed \(n\) razredov. Zaradi biološke smiselnosti mora seveda veljati \(x_i\geq 0\),
\(i=1,2,\ldots,n\). Razredi se med seboj ločijo na okužene in neokužene. 
Naj bo prvih \(m\) razredov okuženih, preostali pa neokuženi. Vpeljimo še oznako
\(X_s=\{x\geq 0 | x_i=0, i=1,\ldots, m\}\), to je množica vseh stanj, v katerih
ni bolezni. \todo{nove okužbe}

Sedaj z \(\F_i(x)\) označimo stopnjo pojavitve novih okužb v razredu \(i\). 
Z \(\V^+_i(x)\) označimo stopnjo prehoda v \(i\)-ti razred, ki se ne zgodijo zaradi 
novih okužb temveč iz drugih razlogov, \(\V^-_i\) pa naj označuje stopnjo prehoda
iz \(i\)-tega razreda. 

Model prenosa okužbe lahko zapišemo na naslednji način:
\begin{equation} \label{eq1}
\dot{x}=f_i(x)=\F_i - \V_i,
\end{equation}
kjer je \(\V_i=\V_i^- - \V_i^+\) in \(i=1,\ldots,n\). 
Pisali bomo tudi \(\F=(\F_1,\F_2,\ldots,\F_n)\) in \(\V_1,\V_2,\ldots,\V_n\).

Dobljeni sistem lineariziramo:
\begin{equation} \label{eq2}
\dot{x}=Df(x_0)(x-x_0),
\end{equation}
kjer z \(x_0\) označimo DFE, z \(Df(x_0)\) pa Jacobijevo matriko preslikave 
\(f=(f_i, f_2, \ldots f_n)\).
Za funkcije \(f_i\)
%\(\F_i\), \(\V_i^+\) in \(\V_i^-\) 
morajo veljati še naslednje predpostavke:

\begin{itemize}
    \item[(A1)] \(x\geq 0 \Rightarrow \F_i\textrm{, }\V_i^+\textrm{, }\V_i^- \geq 0\)
    \item[(A2)] \(x_i=0 \Rightarrow \V_i^- =0\), kar pomeni, da prehodi iz praznega 
    razreda niso možni. \todo{v posebnem} 
    \item[(A3)] \(\F_i=0\) za \(i>m\), kar pomeni, da nimamo okužb ve neokuženih razredih - 
    ko se posameznik okuži, preide v okužen razred 
    \item[(A4)] \(x\in X_s \Rightarrow \F_i(x)=0\) in \(\V_i^+(x)=0\) za \(i=1,\ldots,m\)
    To pomeni, da okužba ne pride "od zunaj" temveč samo iz razredov znotraj populacije. 
    \item[(A5)] \(\F(x)=0 \Rightarrow\) vse lastne vrednosti matrike \(Df(x_0)\) imajo 
    negativne realne dele, torej, omejimo se na sisteme, kjer je DFE stabilno, če nimamo novih okužb.    
\end{itemize}

Za lažje delo s prej definiranimi funkcijami si oglejmo naslednjo lemo.

\begin{lema} \label{lema1}
    Naj bo \(x_0\) DFE sistema \ref{eq1} in naj funkcije \(f_i\) zadoščajo predpostavlkam
    (A1)-(A5). Potem sta Jacobijevi matriki za \(\F\) in \(\V\) oblike
    \begin{align*}
        D\F(x_0)= 
        \begin{bmatrix}
        F & 0 \\
        0 & 0 \\
        \end{bmatrix},\textrm{ }
        D\V(x_0)=
        \begin{bmatrix}
        V & 0 \\
        J_3 & J_4 \\
        \end{bmatrix},
    \end{align*}
    kjer sta \(F\) in \(V\) \(m\times m\) matriki, definirani kot
    \begin{align*}
        \big[\frac{\partial \F_i}{\partial x_j}(x_0)\big]_{i,j=1,\ldots m},\textrm{ }
        \big[\frac{\partial \V_i}{\partial x_j}(x_0)\big]_{i,j=1,\ldots m}.
    \end{align*}
    Velja še, da je \(F\) nenegativna, \(V\) je nesingularna M-matrika in vse lastne
    vrednosti matrike \(J_4\) imajo pozitivne realne dele.
\end{lema}

\begin{opomba}
    Pravimo, da je neka matrika M-matrika, če je Z-matrika in imajo njene lastne
    vrednosti nenegativne realne dele. Matriko imenujemo Z-matrika, če so vsi njeni
    izvendiagonalni elementi nepozitivni.
\end{opomba}

\todo{dokaz}

Sedaj, ko imamo definiran model populacije, lahko definiramo osnovno reprodukcijsko
število. Osnovno reprodukcijsko število \(\R_0\) v epidemiološkem modelu interpretiramo
kot pričakovano število novih okužb, ki jih povzroči okuženi osebek v sicer popolnoma
dovzetni populaciji. Če je \(R_0<1\), torej v povprečju okuženi osebek v času svoje
okuženosti ustvari manj kot eno novo okužbo in okužba se zato ne bo razširila. Podobno, če je 
\(R_0>1\), v povprečju okuženi osebek ustvari več kot enega novega okuženega, kar pomeni,
da se okužba lahko razširi po populaciji. 

Matematično lahko v primeru, da imamo samo en razred okuženih, \(\R_0\) definiramo
kot produkt med stopnjo okužbe in pričakovanim časom, ki ga okuženi osebek preživi 
v okuženem razredu. V modelu z več okuženimi razredi pa \(\R_0\) izpeljemo na naslednji način.

Opazovati moramo, kaj se dogaja z okuženim osebkom, ki ga uvedemo v populacijo brez 
bolezni, torej v populacijo v DFE \(x_0\). Ker v začetku še nimamo okužbe, je 
\(D\F(x_0)\) ničeln, torej opazujemo enačbo
\begin{equation}\label{eq3}
    \dot{x}=-D\V(x_0)(x-x_0)
\end{equation}
Sedaj s \(\Psi_i(0)\) označimo število okuženih, ki jih v začetku uvedemo v \(i\)-ti 
razred. Naj bo \(\Psi(t)=(\Psi_1(t),\ldots,\Psi_m(t))^T\) število okuženih ob času \(t\). 
Po lemi \ref{lema1} se da zapisati \(D\V\) kot 
\(
\begin{bmatrix}
    V & 0 \\
    J_3 & J_4 \\
\end{bmatrix}
\),
kjer je \(V\) velikosti \(m \times m\), vektor \(\Psi\) pa v bistvu
predstavlja prvih \(m\) komponent vektorja \(x\), ki reši enačbo \ref{eq3}. Torej 
\(\Psi(t)\) reši enačbo \(\Psi'(t)=-V\Psi(t)\). Ta enačba ima enolično rešitev 
\(\Psi(t)=\exp{-Vt}\Psi(0)\).

Pričakovano število okužb, ki jih ustvarijo okuženi posamezniki v tem primeru,
se izračuna kot \(\int_0^\infty F\Psi(t)dt = \int_0^\infty F\exp{-Vt}\Psi(0)dt = FV^{-1}\Psi(0)\).
Po lemi \ref{lema1} je matrika \(V\) nesingularna, torej \(V^{-1}\) obstaja. 

Sedaj si oglejmo interpretacijo matrike \(FV^{-1}\). Element na \((i,j)\)-tem mestu
v matriki \(F\) predstavlja stopnjo, s katero okuženi osebki iz razreda \(j\)
ustvarijo nove okužbe v razredu \(i\). Element na \((j,k)\)-tem mestu v matriki \(V^{-1}\)
predstavlja pričakovani čas, ki ga nek okuženi posameznik, ki je začel v \(k\)-tem razredu, 
preživi v \(j\)-tem razredu v času svojega življenja. Torej je \((i,k)\)-ti element 
matrike \(FV^{-1}\) pričakovano število novih okužb, ki jih okuženi posameznik, ki 
smo ga na začetki uvedli v \(k\)-ti razred, povzroči v \(i\)-tem razredu.

Osnovno reprodukcijsko število definiramo kot spektralni radij matrike \(FV^{-1}\):
\[\R_0=\rho(FV^{-1}).\]

O povezavi med osnovnim reprodukcijskim številom in lokalno asimptotsko stabilnostjo 
ravnovesnega stanja DFE govori naslednji izrek.

\begin{izrek}\label{izrek1}
    Imejmo model prenosa bolezni kot v \ref{eq1}, kjer naj za funkcijo \(f\) veljajo
    predpostavke (A1)-(A5). Če je ravnovesno stanje \(x_0\) DFE, potem je \(x_0\) LAS, 
    če velja \(\R_0<1\), in nestabilno, če je \(\R_0>1\), kjer je \(\R_0=\rho(FV^{-1})\).
\end{izrek}

\todo{dokaz}

\section{Podkritična in nadkritična ravnovesna stanja}

Videli smo, da je ravnovesje brez okužbe LAS če je \(\R_0<1\) in nestabilno, če 
je \(\R_0>1\). Težava pa se pojavi v točki bifurkacije, torej če \(\R_0=1\), oziroma
v njeni okolici.

Za lažjo obravnavo dogajanja v okolici točke bifurkacije vpeljimo bifurkacijski parameter
\(\mu\), za katerega naj velja, da je \(\R_0<1\) za \(\mu<0\) in \(\R_0>1\) za \(\mu>0\).

Opazujemo sistem 

\begin{equation}\label{eq4}
\dot{x}=f(x,\mu),
\end{equation}

kjer je \(f\) kot v poglavju \ref{r0} in vsaj dvakrat zvezno odvedljiva v \(x\) in \(\mu\).
Naj ima sistem DFE \(x_0\). Lokalna stabilnost ravnovesja se spremeni v točki \((x_0,0)\), kjer 
imamo torej bifurkacijo. V okolici točke bifurkacije se nam lahko pojavijo endemična 
ravnovesja, ki so lahko superkritična ali subkritična. Superkritična ravnovesja so 
netrivialna \todo{stabilna?} ravnovesja v okolici točke bifurkacije pri \(\R_0>1\),
subkritična pa pri \(\R_0<1\). Slednja lahko pomembno vplivajo na dinamiko v 
epidemioloških modelih, saj lahko že pri majhnih perturbacijah okrog DFE preidemo
v nestabilno ravnovesno stanje in se zato kljub temu, da je \(\R_0<1\), okužba lahko 
razširi v populaciji.

Obstoj nadkritičnih in podkritičnih ravnovesnih stanj pokažemo s pomočjo teorije 
centralne mnogoterosti. Centralno mnogoterost \(W^c\) sestavljajo orbite, za katere velja, da na 
njihovo obnašanje okrog točke ravnovesja ne vpliva niti stabilna mnogoterost \(W^s\), niti nestabilna 
mnogoterost \(W^u\). Stabilna mnogoterost pripada lastnemu prostoru, ki ga razpenjajo lastni vektorji 
lastnih vrednosti z negativnim realnim delom, nestabilna mnogoterost pa lastnemu prostoru, razpetemu z 
lastnimi vektorji, ki pripadajo lastnim vrednostim s pozitivnim realnim delom. Lastnim vrednostim z ničelnim 
realnim delom pripadajo lastni vektorji, ki razpenjajo prostor, ki mu pripada centralna mnogoterost. 

V nadaljevanju bomo potrebovali izrek o centralni mnogoterosti, ki pravi naslednje:
%\begin{izrek}
%    Imejmo nelinearni sistem oblike 
%    \[\dot{x}=Ax+f(x,y)\]
%    \[\dot{y}=By+g(x,y)\],
%    kjer \(f(0,0)=g(0,0)=Df(0,0)=Dg(0,0)=0\), \((x,y)\in \mathbb{R}^c\times \mathbb{R}^s\) 
%    in \(A\) \(c \times c\) matrika z ničelnimi realnimi deli, \(B\) pa matrika z negativnimi realnimi deli, 
%    ter \(f\) in \(g\) vsaj dvakrat zvezno odvedljivi. Potem obstaja 
%    centralna mnogoterost 
%    \[W^c=\{(x,y)|y=h(x), |x|<\delta, h(0)=0, Dh(0)=0\},\]
%    za katero velja, da je v \((x,y)=(0,0)\) tangentna na prostor \(E^c\), ki ga razpenjajo lastni vektorji ničelnih lastnih vrednosti. 
%    Dinamika sistema, omejenega na centralno mnogoterost je podana s sistemom 
%    \(\dot{u}=Au+f(u,h(u))\) za dovolj majhen \(u\in \mathbb{R}^c\).
%\end{izrek}
\begin{izrek}
    Imejmo nelinearni sistem \(\dot{x}=f(x),\textrm{ }x\in\mathbb{R}^n\), naj bo \(f\) gladka 
    in \(x=0\) stacionarna točka. Naj ima \(Df(0)\) lastne vrednosti s pozitivnimi, negativnimi in 
    ničelnimi realnimi deli. Pripadajoči lastni vektorji razpenjajo prostore \(E^s\), \(E^u\) in \(E^c\). 
    Potem obstajata stabilna mnogoterost \(W^s\), enake dimenzije kot \(E^c\) in tangentna na \(E^c\) v \(x=0\) in nestabilna mnogoterost \(W^u\),
    enake dimenzije kot \(E^u\) in tangentna nanj v \(x=0\), ter invariantna centralna mnogoterost \(W^c\), 
    tangentna na \(E^c\) v \(x=0\).{}
\end{izrek}

Imejmo sedaj matriko \(D_xf(x_0,0)\) (Jacobijeva matrika odvodov po komponentah \(x\)-a). 
Ker imamo v \((x_0,0)\), bifurkacijo, je \(0\) lastna vrednost te matrike. Naj bo 
to enostavna lastna vrednost matrike. Naj bosta vektorja \(v\) in \(w\) taka, da je 
\(vD_xf(x_0,0)=0\) in \(D_xf(x_0,0)u=0\) in da \(vu=1\).
%Po lemi \ref{lema1} in izreku \ref{izrek1} imajo vse ostale lastne vrednosti negativne realne dele.
Naj bo 
\begin{equation}\label{eq5}
a=\frac{v}{2}D_{xx}f(x_0,0)w^2=\frac{1}{2}\sum_{i,j,k=1}^n v_iw_jw_k\frac{\partial^2f_i}{\partial x_j \partial x_k}(x_0,0)
\end{equation}
\[b=vD_{x\mu}f(x_0,0)w=\sum_{i,j=1}^n v_i w_j \frac{\partial^2 f_i}{\partial x_j \partial \mu}(x_0,0).\]
Izraz za \(a\) je možno zapisati še na drugačen način. Velja naslednje:

\begin{lema}\label{lema2}
    Naj bo \(f(x,\mu)\) vsaj dvakrat zvezno odvedljiva v 
    \(x\) in \(\mu\) in naj zanjo veljajo predpostavke (A1)-(A5).
    Naj bo \(0\) enostavna lastna vrednost \(D_xf(x_0,0)\) in \(v\) in \(w\)
    vektorja, za katera \(vD_xf(x_0,0)=0\) in \(D_xf(x_0,0)w=0\). 
    Potem \(v_i\geq 0\) in \(w_i\geq 0\) za \(i=1,\ldots, m\) in 
    \(v_i=0\) za \(i=m+1,\ldots,n\) ter 
    \[a=\sum_{i,j,k=1}^m v_i w_j w_k \big(\frac{1}{2}\frac{\partial^2 f_i}{\partial x_j \partial x_k}(x_0,0)+\sum_{l=m+1}^n \alpha_{lk}\frac{\partial^2 f_i}{\partial x_j \partial x_l}(x_0,0)\big),\]
    kjer \(\alpha_{lk}\) (\(l=m+1,\ldots,n,\textrm{ }k=1,\ldots,m\)) označuje 
    \((l-m,k)\)-ti element matrike \(-J_4^{-1}J_3\), kjer sta matriki \(J_3\) in \(J_4\)
    kot v lemi \ref{lema1}.
\end{lema}

\todo{dokaz:izpusti}

Vrednost \(a\) odloča o značilnostih endemičnih ravnovesij v bližini točke bifurkacije, 
torej o tem, ali imamo superkritična ali subkritična ravnovesja. 

\begin{izrek}\label{izrek2}
    Imejmo sistem \ref{eq4}, kjer za \(f\) velja (A1)-(A5). Naj bo \(0\) enostavna
    lastna vrednost \(D_xf(x_0,0)\). Naj bo \(a\) kot v \label{eq5} in naj velja \(b\neq 0\).
    Potem obstaja \(\delta >0\), da velja:
    \begin{itemize}
        \item če \(a<0\), potem obstajajo lokalno asimptotsko stabilna endemična ravnovesja
        v bližini \(x_0\) za \(0<\mu <\delta\) (superkritična ravnovesja)
        \item če \(a>0\), potem obstajajo nestabilna endemična ravnovesja v bližini \(x_0\) 
        za \(-\delta < \mu < 0\).
    \end{itemize}
\end{izrek}

\todo{dokaz}

\section{Primeri}
\subsection{Model zdravljenja}

Model, s katerim se ukvarjamo v tem podpoglavju, temelji na nekaterih modelih 
za tuberkulozo. Populacijo razdelimo v štiri razrede: razred dovzetnih \(S\), 
razred izpostavljenjih \(E\), razred okuženih \(I\) in razred zdravljenih \(T\). 
Velikost celotne populacije označimo z \(N\), torej je \(S+E+I+T=N\). Razreda \(E\) 
in \(I\) smatramo kot okužena razreda, preostala dva pa kot neokužena. 
Model predstavimo z naslednjimi enačbami
\begin{align*}
&\dot{E}=\beta_1\frac{SI}{N}+\beta_2\frac{TI}{N}-(d+\nu+r_1)E+pr_2I,\\
&\dot{I}=\nu E -(d+r_2)I,\\
&\dot{S}=b(N)-dS-\beta_1\frac{SI}{N},\\
&\dot{T}=-dT+r_1E+qr_2I-\beta_2\frac{TI}{N}.
\end{align*}
Tu je \(\beta_1\frac{I}{N}\) stopnja, s katero dovzetni postanejo izpostavljeni, 
\(\beta_2\frac{I}{N}\) pa stopnja, s katero zdravljeni postanejo izpostavljeni. 
Izpostavljeni postanejo okuženi s stopnjo \(\nu\). Stopnja zdravljenja izpostavljenih 
je \(r_1\), stopnja zdravljenja okuženih pa \(r_2\). Zdravljenje okuženih ni vedno uspešno. 
S \(q\) označimo delež uspešnih zdravljenj okuženih posameznikov, medtem ko delež \(p=1-q\) 
okuženih, ki so bili zdravljeni, ne ozdravi, temveč preide nazaj v razred izpostavljenih. 
Z \(b(N)\) označimo stopnjo rodnosti, \(d\) pa je stopnja smrtnosti za vse razrede. Vsi novorojeni 
so na začetku dovzetni. 

Analize modela se najprej lotimo tako, da določimo vektorja 
\(\F\) in \(\V\), torej da določimo, kateri prehodi med razredi so nove okužbe in kateri ne. 
Prehodov med \(E\) in \(I\) ne razumemo kot nove okužbe. Nova okužba je le ali prehod iz \(S\) v \(E\) 
ali iz \(T\) v \(E\). Zapišimo torej vektorja \(\F\) in \(\V\):
\begin{align*}
    \F= 
        \begin{bmatrix}
        \beta_1\frac{SI}{N}+\beta_2\frac{TI}{N}\\
        0\\
        0\\
        0\\
        \end{bmatrix},\textrm{ }
        \V= 
        \begin{bmatrix}
        (d+\nu+r_1)E-pr_2I \\
        -\nu E +(d+r_2)I\\
        -b(N)+dS+\beta_1 \frac{SI}{N}\\
        dT-r_1E-qr_2I+\beta_2\frac{TI}{N}
        \end{bmatrix}.
\end{align*}
Tu smatramo \(E\) kot prvi razred, \(I\) kot drugi, \(S\) tretji in \(T\) četrti. 
Ravnovesje brez okužbe (DFE) je tisto, pri katerem je \(E=0\) in \(I=0\). 
V tem primeru dobimo enačbe 
\begin{align*}
    &\dot{E}=0\\
    &\dot{I}=0\\
    &\dot{S}=b(N)-dS\\
    &\dot{T}=-dT.
\end{align*}
Sledi, da mora biti v tem ravnovesju \(-dT=0\), torej \(T=0\), in 
\(b(N)-dS=0\), torej je \(S=S_0\), za katerega velja \(b(S_0)=dS_0\). 
Dobimo torej ravnovesje DFE \(x_0=(0,0,S_0,0)^T\). 
Predpostavimo, da je \(S_0=1\). 

Zapišimo še matriki \(F\) in \(V\) iz zapisa v lemi \ref{lema1}. Imamo \(2\) okužena razreda, torej \(m=2\). 
Torej \(F\) in \(V\) vsebujeta 
parcialne odvode \(\F_i\) in \(V_i\) po \(E\) in \(I\) evalvirane v ravnovesju \(x_0\). 
Velja na primer \(\frac{\partial \F_1}{\partial E}|_{x_0}=0\) in 
\(\frac{\partial \F_1}{\partial I}|_{x_0}=\beta_1\). Podobno naredimo še za ostale komponente \(\F\) in \(\V\) 
in dobimo 
\begin{align*}
    F= 
    \begin{bmatrix}
    0 & \beta_q \\
    0 & 0 \\
    \end{bmatrix},\textrm{ }
    V=
        \begin{bmatrix}
        d+\nu+r_1 & -pr_2 \\
        -\nu & d+r_2 \\
        \end{bmatrix},\textrm{ }
\end{align*}
Za izračun \(\R_0\) moramo izračunati še \(V^{-1}\):
\begin{align*}
    V^{-1}=\frac{1}{det(V)}
    \begin{bmatrix}
    d+r_2 & pr_2 \\
    \nu & d+\nu+r_1 \\
    \end{bmatrix}=
    \frac{1}{(d+\nu+r_1)(d+r_2)-\nu pr_2}
    \begin{bmatrix}
        d+r_2 & pr_2 \\
        \nu & d+\nu+r_1 \\
    \end{bmatrix}.
\end{align*}

Sedaj lahko izračunamo \(\R_0=FV^{-1}=\frac{\beta_1\nu}{(d+\nu+r_1)(d+r_2)-\nu pr_2}\) 
in s tem smo dobili pogoj, ki pove, ali je ravnovesje \(x_0\) LAS ali nestabilno.

Osnovno reprodukcijsko število \(\R_0\) lahko v tem primeru dobimo tudi na nekoliko 
drugačen način. Poskusimo interpretirati, kaj pomeni element v drugi vrstici in prvem stolpcu 
matrike \(V^{-1}\) (ta je namreč edini, ki se pri množenju z \(F\) ne bo množil z \(0\)). 
Ta pomeni pričakovani čas, ki ga osebek iz prvega razreda, torej iz \(E\), preživi v drugem razredu, torej v \(I\). 
Označimo s \(h_1=\frac{\nu}{d+\nu+r_1}\) delež osebkov iz \(E\), ki preidejo v \(I\), in s 
\(h_2=\frac{pr_2}{d+r_2}\) delež osebkov iz \(I\), ki preidejo nazaj v \(E\). 
Delež \(h_1\) osebkov iz \(E\) gre v \(I\) vsaj enkrat, delež \(h_1^2h_2\) gre v \(I\) vsaj dvakrat, 
delež \(h_1^{k}h_2^{k-1}\) gre v \(I\) vsaj \(k\)-krat. 
Vsak tak osebek vsakič v \(I\) preživi v povprečju \(\frac{1}{d+r_2}\) časa. Torej je pričakovani čas, ki ga 
posameznik iz razreda \(E\) preživi v \(I\) enak 
\[\frac{1}{d+r_2}(h_1+h_1^2h_2+\cdots)=\frac{1}{d+r_2}\frac{h_1}{1-h_1h_2}=\frac{\nu}{(d+\nu+r_1)(d+r_2)-\nu pr_2}.\]
Če to pomnožimo z \(\beta_1\), dobimo ravno \(\R_0\), kakršnega smo izpeljali zgoraj.

Oglejmo si še, kaj se v tem biološkem modelu dogaja okrog točke bifurkacije. 
Videti se da, da je pri \(\R_0=1\) \(0\) res enostavna lastna vrednost matrike 
\(D_xf(x_0,0)\), torej lahko uporabimo izrek \ref{izrek2}. Za izračun \(a\) potrebujemo 
druge odvode \(f_i\) v \(E\) in \(I\) v ravnovesju. Opazimo lahko, da so edini neničelni drugi odvodi
\(\frac{\partial^2f_1}{\partial E \partial I}=-\beta_1\), 
\(\frac{\partial^2f_1}{\partial I^2}=-2\beta_1\) in 
\(\frac{\partial^2f_1}{\partial I \partial T}=\beta_2-\beta_1\). 
Torej je \[a=-\beta_1 v_1 w_2(w_1+w_2+(1-\frac{\beta_2}{\beta_1})w_4).\]
Vektorja \(v\) in \(w\) lahko izberemo tako, da so vse komponente \(w\) pozitivne 
in \(v_1\) pozitiven. Ker je biološko smiselno, da je \(\beta_2<\beta_1\), je \(a<0\). 
Torej je DFE \(x_0\) LAS če je \(\R_0\) nekoliko manjši od 1, če pa je \(\R_0\) nekoliko večji od 1,
je DFE nestabilno, obstaja pa LAS ravnovesje blizu DFE. 

Obstaja pa še nekoliko drugačna variacija tega modela, kjer dodamo 
nek nov prehod med \(E\) in \(I\) in je torej 
\(\dot{I}=\nu E -(d+r_2)I +\beta_3 \frac{EI}{N}\) in \(\dot{E}= \beta_1\frac{SI}{N}+\beta_2\frac{TI}{N}-(d+\nu+r_1)E+pr_2I-\beta_3 \frac{EI}{N}\).
V tem primeru dobimo enako ravnovesje DFE in enak \(\R_0\), za \(a\) pa dobimo 
\[a=-\beta_1v_1w_2(w_1+w_2+(1-\frac{\beta_2}{\beta_1})w_4)+\beta_3w_1w_2(v_2-v_1).\]
Pokazati se da, da je \(v_2-v_1>0\). Če je \(\beta_3\) dovolj velik, bo \(a>0\). 
Torej bo obstajalo nestabilno podkritično ravnovesje blizu DFE. Majhne perturbacije v populaciji 
torej lahko povzročijo razširjanje bolezni, kljub temu, da je DFE stabilno.
\end{document}