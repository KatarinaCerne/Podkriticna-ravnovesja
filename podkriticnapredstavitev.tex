\documentclass[11pt]{beamer}

\usetheme{Berlin}
\usecolortheme{seahorse}

\usepackage[slovene]{babel}
\usepackage[utf8]{inputenc}
\usepackage{lmodern}
\usepackage[T1]{fontenc}
\usepackage{eurosym}
\usepackage{graphicx}
\usepackage{amsfonts}
\usepackage{amssymb}
\usepackage{amsmath}
\usepackage{hyperref}
\usepackage{amsthm}
\usepackage{subfigure}
\usepackage{xcolor}
\usepackage{tcolorbox}
%\usepackage{enumitem}

\setbeamercovered{invisible}

\newcommand{\todo}[1]{{\color{red}{#1}}}
\newcommand{\R}{\mathcal R}
\newcommand{\F}{\mathcal F}
\newcommand{\V}{\mathcal V}

\newcommand{\myitem}{\item[\bullet]}

\newtheorem{lema}{Lema}
\newtheorem{izrek}{Izrek}
\newtheorem{opomba}{Opomba}

\title{Osnovna reprodukcijska števila in podkritična endemična ravnovesna stanja v epidemioloških problemih}
\author{Katarina Černe}

\begin{document}

\begin{frame}
\maketitle
\end{frame}

\section{Uvod}
\begin{frame}
\begin{itemize}
    \item pogosto je eno izmed ravnovesnih stanj ravnovesje brez okužbe (DFE)
    \item pri analizi si pomagamo z osnovnim reprodukcijskim številom \(\R_0\)
    \item če je \(\R_0<1\), je DFE LAS
    \item če je \(\R_0>1\), je DFE nestabilno
    \item pogoj \(\R_0<1\) ne zagotavlja nujno, da lahko bolezen izkoreninimo
\end{itemize}
\end{frame}

\section{Osnovno reprodukcijsko število v razdrobljenih epidemioloških problemih}
\begin{frame}
\begin{itemize}
    \item populacija z \(n\) razredi \(x=(x_1,x_2,\ldots,x_n)^T\), \(x_i\geq 0\)
    \item prvih \(m\) razredov okuženih, preostali neokuženi
    \item \(X_s=\{x\geq 0 | x_i=0, i=1,\ldots, m\}\) množica vseh stanj, v katerih
    ni bolezni
\end{itemize}

\begin{itemize}
    \item \(\F_i(x)\) stopnja pojavitve novih okužb v razredu \(i\)
    \item \(\V^+_i(x)\) stopnja prehoda v \(i\)-ti razred, ki se ne zgodijo zaradi 
    novih okužb
    \item \(\V^-_i\) stopnja prehoda iz \(i\)-tega razreda 
\end{itemize}
\end{frame}

\begin{frame}
    Model prenosa okužbe:
    \begin{equation} \label{eq1}
        \dot{x_i}=f_i(x)=\F_i - \V_i,
    \end{equation}
    \(\V_i=\V_i^- - \V_i^+\) in \(i=1,\ldots,n\)\\
    Linearizacija:
    \begin{equation} \label{eq2}
        \dot{x}=Df(x_0)(x-x_0),
    \end{equation}
\end{frame}

\begin{frame}
    Za funkcije \(f_i\) morajo veljati še naslednje predpostavke:
    \begin{itemize}
        \item[(A1)] \(x\geq 0 \Rightarrow \F_i\textrm{, }\V_i^+\textrm{, }\V_i^- \geq 0\)
        \item[(A2)] \(x_i=0 \Rightarrow \V_i^- =0\), kar pomeni, da prehodi iz praznega 
        razreda niso možni. 
        \item[(A3)] \(\F_i=0\) za \(i>m\), kar pomeni, da nimamo okužb ve neokuženih razredih - 
        ko se posameznik okuži, preide v okužen razred 
        \item[(A4)] \(x\in X_s \Rightarrow \F_i(x)=0\) in \(\V_i^+(x)=0\) za \(i=1,\ldots,m\)
        To pomeni, da okužba ne pride "od zunaj" temveč samo iz razredov znotraj populacije. 
        \item[(A5)] \(\F(x)=0 \Rightarrow\) vse lastne vrednosti matrike \(Df(x_0)\) imajo 
        negativne realne dele, torej, omejimo se na sisteme, kjer je DFE stabilno, če nimamo novih okužb.    
    \end{itemize}
\end{frame}

\begin{frame}
    \begin{lema} \label{lema1}
        Naj bo \(x_0\) DFE sistema \ref{eq1} in naj funkcije \(f_i\) zadoščajo predpostavlkam
        (A1)-(A5). Potem sta Jacobijevi matriki za \(\F\) in \(\V\) oblike
        \begin{align*}
            D\F(x_0)= 
            \begin{bmatrix}
            F & 0 \\
            0 & 0 \\
            \end{bmatrix},\textrm{ }
            D\V(x_0)=
            \begin{bmatrix}
            V & 0 \\
            J_3 & J_4 \\
            \end{bmatrix},
        \end{align*}
        kjer sta \(F\) in \(V\) \(m\times m\) matriki, definirani kot
        \begin{align*}
            \big[\frac{\partial \F_i}{\partial x_j}(x_0)\big]_{i,j=1,\ldots m},\textrm{ }
            \big[\frac{\partial \V_i}{\partial x_j}(x_0)\big]_{i,j=1,\ldots m}.
        \end{align*}
        Velja še, da je \(F\) nenegativna, \(V\) je nesingularna M-matrika in vse lastne
        vrednosti matrike \(J_4\) imajo pozitivne realne dele.
    \end{lema}
\end{frame}

%tu povej, kaj je M-matrika

\begin{frame}
    Osnovno reprodukcijsko število \(\R_0\) v epidemiološkem modelu = pričakovano število novih okužb, ki jih povzroči okuženi osebek v sicer popolnoma
    dovzetni populaciji
\end{frame}

\begin{frame}
    \(\dot{x}=-D\V(x_0)(x-x_0)\)\\
    \(\Psi_i(0)\) = število okuženih, ki jih v začetku uvedemo v \(i\)-ti razred\\
    \(\Psi(t)=(\Psi_1(t),\ldots,\Psi_m(t))^T\) število okuženih ob času \(t\)\\
    \(\Psi(t)\) reši enačbo \(\Psi'(t)=-V\Psi(t)\)\\
    \(\Psi(t)=e^{-Vt}\Psi(0)\)\\

    Pričakovano število okužb, ki jih ustvarijo okuženi posamezniki = \\
    \(\int_0^\infty F\Psi(t)dt = \int_0^\infty Fe^{-Vt}\Psi(0)dt = FV^{-1}\Psi(0)\)
\end{frame}

\begin{frame}
    \begin{itemize}
    \item \((i,j)\)-ti element \(F\): stopnja, s katero okuženi osebki iz razreda \(j\)
    ustvarijo nove okužbe v razredu \(i\)
    \item \((j,k)\)-ti element \(V^{-1}\): pričakovani čas, ki ga okuženi posameznik, ki je začel v \(k\)-tem razredu, 
    preživi v \(j\)-tem razredu v času svojega življenja
    \item \((i,k)\)-ti element  \(FV^{-1}\): pričakovano število novih okužb, ki jih okuženi posameznik, ki 
    smo ga na začetki uvedli v \(k\)-ti razred, povzroči v \(i\)-tem razredu
    \end{itemize}
    \[\R_0=\rho(FV^{-1})\]
\end{frame}

\begin{frame}
    \begin{izrek}\label{izrek1}
        Imejmo model prenosa bolezni kot v \ref{eq1}, kjer naj za funkcijo \(f\) veljajo
        predpostavke (A1)-(A5). Če je ravnovesno stanje \(x_0\) DFE, potem je \(x_0\) LAS, 
        če velja \(\R_0<1\), in nestabilno, če je \(\R_0>1\), kjer je \(\R_0=\rho(FV^{-1})\).
    \end{izrek}
\end{frame}

%tu pride primer?

\section{Podkritična in nadkritična ravnovesna stanja}

\begin{frame}
    Težava v točki bifurkacije, torej če \(\R_0=1\), oziroma v njeni okolici

    Opazujemo sistem 
    \begin{equation}\label{eq4}
    \dot{x}=f(x,\mu),
    \end{equation}

    bifurkacijski parameter \(\mu\): \(\R_0<1\) za \(\mu<0\) in \(\R_0>1\) za \(\mu>0\)\\

    bifurkacija v točki \((x_0,0)\)
\end{frame}

\begin{frame}
    \begin{itemize}
        \item v okolici točke bifurkacije se lahko pojavijo endemična 
        ravnovesja: nadkritična ali podkritična
        \item nadkritična: netrivialna ravnovesja v okolici točke bifurkacije pri \(\R_0>1\)
        \item podkritična: netrivialna ravnovesja v okolici točke bifurkacije pri \(\R_0<1\)
    \end{itemize}
\end{frame}

\begin{frame}
    \(0\) enostavna lastna vrednost matrike \(D_xf(x_0,0)\)\\
    \(v\) in \(w\) levi in desni lastni vektor, \(vw=1\)\\
    \begin{equation}\label{eq5}
        a=\frac{v}{2}D_{xx}f(x_0,0)w^2=\frac{1}{2}\sum_{i,j,k=1}^n v_iw_jw_k\frac{\partial^2f_i}{\partial x_j \partial x_k}(x_0,0)
        \end{equation}
        \[b=vD_{x\mu}f(x_0,0)w=\sum_{i,j=1}^n v_i w_j \frac{\partial^2 f_i}{\partial x_j \partial \mu}(x_0,0).\]
\end{frame}

\begin{frame}
    \begin{lema}\label{lema2}
        Naj bo \(f(x,\mu)\) vsaj dvakrat zvezno odvedljiva v 
        \(x\) in \(\mu\) in naj zanjo veljajo predpostavke (A1)-(A5).
        Naj bo \(0\) enostavna lastna vrednost \(D_xf(x_0,0)\) in \(v\) in \(w\)
        vektorja, za katera \(vD_xf(x_0,0)=0\) in \(D_xf(x_0,0)w=0\). 
        Potem \(v_i\geq 0\) in \(w_i\geq 0\) za \(i=1,\ldots, m\) in 
        \(v_i=0\) za \(i=m+1,\ldots,n\) ter 
        \[a=\sum_{i,j,k=1}^m v_i w_j w_k \big(\frac{1}{2}\frac{\partial^2 f_i}{\partial x_j \partial x_k}(x_0,0)+\sum_{l=m+1}^n \alpha_{lk}\frac{\partial^2 f_i}{\partial x_j \partial x_l}(x_0,0)\big),\]
        kjer \(\alpha_{lk}\) (\(l=m+1,\ldots,n,\textrm{ }k=1,\ldots,m\)) označuje 
        \((l-m,k)\)-ti element matrike \(-J_4^{-1}J_3\), kjer sta matriki \(J_3\) in \(J_4\)
        kot v lemi \ref{lema1}.
    \end{lema}
\end{frame}

\begin{frame}
    \begin{izrek}\label{izrek2}
        Imejmo sistem \ref{eq4}, kjer za \(f\) velja (A1)-(A5). Naj bo \(0\) enostavna
        lastna vrednost \(D_xf(x_0,0)\). Naj bo \(a\) kot zgoraj in naj velja \(b\neq 0\).
        Potem obstaja \(\delta >0\), da velja:
        \begin{itemize}
            \item če \(a<0\), potem obstajajo lokalno asimptotsko stabilna endemična ravnovesja
            v bližini \(x_0\) za \(0<\mu <\delta\) (nadkritična ravnovesja)
            \item če \(a>0\), potem obstajajo nestabilna endemična ravnovesja v bližini \(x_0\) 
            za \(-\delta < \mu < 0\) (podkritična ravnovesja).
        \end{itemize}
    \end{izrek}
\end{frame}

\begin{frame}
    \begin{izrek}\label{izrekcm}
        Imejmo nelinearni sistem \(\dot{x}=f(x,\mu),\textrm{ }x\in\mathbb{R}^n\), naj bo \(f\) gladka 
        in \((x,\mu)=(0,0)\) stacionarna točka. Naj ima \(Df(0,0)\) lastne vrednosti s pozitivnimi, negativnimi in 
        ničelnimi realnimi deli. Pripadajoči lastni vektorji razpenjajo prostore \(E^s\), \(E^u\) in \(E^c\). 
        Potem obstajata stabilna mnogoterost \(W^s\), enake dimenzije kot \(E^c\) in tangentna na \(E^c\) v \((x,\mu)=(0,0)\) in nestabilna mnogoterost \(W^u\),
        enake dimenzije kot \(E^u\) in tangentna nanj v \((x,\mu)=(0,0)\), ter invariantna centralna mnogoterost \(W^c\), 
        tangentna na \(E^c\) v \((x,\mu)=(0,0)\).
    \end{izrek}
\end{frame}

\begin{frame}
    \[\dot{x}=Ax+f_1(x,y,\mu)\]
\[\dot{y}=By+f_2(x,y,\mu)\],
\[\dot{\mu}=0\],
kjer \(f_1(0,0,0)=f_2(0,0,0)=Df_1(0,0,0)=Df_2(0,0,0)=0\), \((x,y)\in \mathbb{R}^c\times \mathbb{R}^s\)\\
\(A\) \(c \times c\) matrika z ničelnimi realnimi deli, \(B\) z negativnimi realnimi deli\\
obstaja centralna mnogoterost oblike
\[W^c=\{(x,y,\mu)|y=h(x,\mu), |x|<\delta, |\mu|<\delta, h(0,0)=0, Dh(0,0)=0\},\]
Dinamika sistema, omejenega na centralno mnogoterost je podana s sistemom 
\(\dot{u}=Au+f_1(u,h(u,\mu),\mu)\)
\end{frame}

\section{Primeri}
\begin{frame}
    \begin{align*}
        &\dot{E}=\beta_1\frac{SI}{N}+\beta_2\frac{TI}{N}-(d+\nu+r_1)E+pr_2I,\\
        &\dot{I}=\nu E -(d+r_2)I,\\
        &\dot{S}=b(N)-dS-\beta_1\frac{SI}{N},\\
        &\dot{T}=-dT+r_1E+qr_2I-\beta_2\frac{TI}{N}.
        \end{align*}
\end{frame}

\begin{frame}
    \(\R_0=\rho(FV^{-1})=\frac{\beta_1\nu}{(d+\nu+r_1)(d+r_2)-\nu pr_2}\)
\end{frame}

\begin{frame}
    \(h_1=\frac{\nu}{d+\nu+r_1}\) delež osebkov iz \(E\), ki preidejo v \(I\)\\ 
    \(h_2=\frac{pr_2}{d+r_2}\) delež osebkov iz \(I\), ki preidejo nazaj v \(E\)\\
    delež \(h_1^{k}h_2^{k-1}\) gre iz \(E\) v \(I\) vsaj \(k\)-krat\\
    pričakovani čas, ki ga posameznik iz razreda \(E\) preživi v \(I\):
    \[\frac{1}{d+r_2}(h_1+h_1^2h_2+\cdots)=\frac{1}{d+r_2}\frac{h_1}{1-h_1h_2}=\frac{\nu}{(d+\nu+r_1)(d+r_2)-\nu pr_2}.\]
\end{frame}

\begin{frame}
Dogajanje okrog točke bifurkacije:\\
\(a=-\beta_1 v_1 w_2(w_1+w_2+(1-\frac{\beta_2}{\beta_1})w_4)\)
\end{frame}

\begin{frame}
    \(\dot{I}=\nu E -(d+r_2)I +\beta_3 \frac{EI}{N}\)\\
    \(\dot{E}= \beta_1\frac{SI}{N}+\beta_2\frac{TI}{N}-(d+\nu+r_1)E+pr_2I-\beta_3 \frac{EI}{N}\)\\
    \(a=-\beta_1v_1w_2(w_1+w_2+(1-\frac{\beta_2}{\beta_1})w_4)+\beta_3w_1w_2(v_2-v_1)\)
\end{frame}
\end{document}